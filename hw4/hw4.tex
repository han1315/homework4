\documentclass{ctexart}

\usepackage{graphicx}
\usepackage{amsmath}
\usepackage{float}

\title{曼德博集合的生成和探索}

\author{洪晨瀚 \\ 信息与计算科学 3200300133}

\begin{document}

\maketitle
\bibliographystyle{plain}
\graphicspath{{image/}}


\section{摘要}
  曼德博集合是一种在复平面上组成的点的集合,曼德博集合无论放的多大都有不同的细节在内,本文通过绘制曼德博集合的图片来观察其轨迹以及规律。

\section{引言}
曼德博集合是一种在复平面上组成分形的点的集合,以数学家本华曼德博的名字命名。将曼德博集合无限放大都能够有精妙的细节在内,而这瑰丽的图案仅仅由一个简单的公式生成。因此有人认为曼德博集合是``人类有史以来做出的最奇异、最瑰丽的几何图形'',曾被称为``上帝的指纹''。 \cite{branner1989mandelbrot}

\section{问题背景}
曼德博集合是由迭代产生,而迭代是不断重复的过程。在数学上,该过程往往是指计算某个数学函数。就博德曼集合而言,被迭代的是二次多项式$f(x)=x^2+c$,其中c为常量。通过迭代该函数而产生的轨迹,具有极大的研究意义,对迭代理论的研究影响深远。 \cite{shishikura1998hausdorff}

\section{数学理论}
\subsection{定义}
\begin{flushleft}
  迭代函数 : $f(x)=x^2+c$,其中$c$为常量\\
  从$z=z_0$开始迭代 :  \\
  $x_1=x_0+c$ ,$x_0$是初值 \\
  $x_2=x_1+c$ \\
  $\vdots$ \\
  $x_{n-1}=x_{n-2}+c$ \\
  $x_n=x_{n-1}+c$
\end{flushleft}

\subsection{定理}
\begin{flushleft}
  定理一:若 $|c| \leq \frac{1}{4}$,则$c \in M$。 \\
  定理二:若 $c \in M$,则$|c| \leq 2$。 \\
  定理三:若 $c \in M$,则$|z_{n}| \leq 2$。 
\end{flushleft}

\section{算法}
\begin{verbatim}
  for each c in complex ,z=0,count=0
  do z=z^2+c,count+=1
  loop until abs(z)>2 || count>maxcount
  if count>maxcount ,draw c,black
  else draw c,white
\end{verbatim}

\section{数值算例}
\begin{figure}[H]
  \centering
  \begin{minipage}[t]{0.48\textwidth}
    \centering
    \includegraphics[width=6cm]{00}
    \caption{Iteration 100}
  \end{minipage}
  \begin{minipage}[t]{0.48\textwidth}
    \centering
    \includegraphics[width=6cm]{100}
    \caption{Iteration 100 enlarge}
  \end{minipage}
\end{figure}

\begin{figure}[H]
  \centering
  \begin{minipage}[t]{0.48\textwidth}
    \centering
    \includegraphics[width=6cm]{01}
    \caption{Iteration 1000}
  \end{minipage}
  \begin{minipage}[t]{0.48\textwidth}
    \centering
    \includegraphics[width=6cm]{1000}
    \caption{Iteration 1000 enlarge}
  \end{minipage}
\end{figure}


\section{结论}
曼德博集合关于$x$轴对称,与$x$轴的交集位于区间$-2$到$\frac{1}{4}$之内。$x$轴上的原点位于主心形内,$-1$点位于主心形左侧的球形内。曼德博集合由所有满足一定条件的复数$c$组成的迭代产生的轨迹不趋于无穷大。

\bibliography{han}
\end{document}

